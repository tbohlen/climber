\section{Conclusion}
\label{sec:conclusion}

Two major learnings from this project are to be careful with the construction of
any hardware components and, of course, to make sure that all registers have the
correct length and are appropriately signed.

Connections between the motors, flex sensors, and wires on each glove came loose
and broke very easily, and the flex sensors in particular deteriorated over
time. Any project using flexible gloves and sensors like this
one needs to be aware of the wear and tear endured by any components attached
directly to the glove.

Overall, though, this project was a success. The use of colored gloves and an
NTSC camera coupled with a high-quality tracking algorithm provided the smooth
hand tracking that the system relied on. Erratic hand motion would have resulted
in an unacceptably jittery screen because of the direct tie between hand
position and screen output.

The gameplay module used the available inputs to display a set of handholds to
the screen and allowed the player to "climb" the virtual wall by closing their
hand while hovering over a hold and moving their hand downward. It provided
simple haptic feedback based on whether or not each hand was in contact with a
hold, but was chosen to be removed due to power supply issues.

